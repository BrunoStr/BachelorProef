%---------- Inleiding ---------------------------------------------------------

\section{Introductie} % The \section*{} command stops section numbering
\label{sec:introductie}

In dit onderzoek gaan we na of virtuele realiteit als hulpmiddel kan fungeren bij de revalidatie van kinesitherapie patiënten en welke mogelijke voordelen het zou hebben. Vele patiënten voeren hun revalidatie oefeningen uit met tegenzin omdat ze vaak saai en repetitief kunnen zijn. Een gevolg hiervan is dat sommigen wel eens een oefensessie overslaan. Een heus probleem in de revalidatiesector dus dat we trachten te onderzoeken en mogelijks kunnen verbeteren of zelfs oplossen.


%---------- Stand van zaken ---------------------------------------------------

\section{State of the art}
\label{sec:start-of-the-art}

\subsection{Literatuurstudie}
Tijdens mijn literatuurstudie stootte ik vooral op artikels deels gelijkaardig aan mijn onderwerp, nl. over het gebruik van virtuele realiteit bij revalidatie na een beroerte. Vele artikels hierover waren echter reeds verouderd. Een interessant artikel dat ik vond was dat van \textcite{Laver2017}. Hier werd een behandeling op basis van virtuele realiteit vergeleken met de traditionele behandeling na een beroerte. In dit onderzoek keek men naar drie verschillende criteria: de armfunctie, de loopsnelheid en de mate waarin de patiënt zelfstandig alledaagse activiteiten kon voltooien. Er werden echter geen significante verschillen aangetroffen tussen beide methoden. Vervolgens vond ik ook een artikel (\textcite{Reddy2018}) dat zich handelt over hoe virtuele realiteit momenteel al vaak wordt ingezet in de psychiatrie om angststoornissen te bestrijden. Hier wordt de patiënt stapsgewijs aan zijn angst blootgesteld in een veilige omgeving en leert hij er langzamerhand mee omgaan. Er gebeurden in het verleden dus al enkele gelijkaardige onderzoeken naar virtuele realiteit in de gezondheidszorg maar nog niet bepaald in de richting waar ik op wil gaan.

\subsection{Bestaande applicaties}
Op mijn zoektoch in het onderzoeksdomein leerde ik een bedrijf kennen genaamd KineQuantum. Dit is een bedrijf dat zich specialiseert in het maken van een VR-applicatie die patiënten bijstaat bij het revalideren. In deze applicatie zitten de patiënten volledig in een virtuele wereld waar ze allerlei oefeningen moeten uitvoeren terwijl de kinesist nog steeds de mogelijkheid heeft om de patiënt op te volgen. Ook wordt de vooruitgang automatisch opgeslagen en deze kan ook geraadpleegd worden door de dokter.
%---------- Methodologie ------------------------------------------------------
\section{Methodologie}
\label{sec:methodologie}

In de onderzoeksfase zal ik nagaan welk VR platform het meest geschikt zou zijn voor het onderzoek. Hierna zal ik een applicatie ontwikkelen voor het gekozen platform waarin de gebruiker op een interactieve, speelse manier bepaalde handelingen moet uitvoeren. In de praktijk zal ik de applicatie vervolgens laten testen door meerdere patiënten. Na de demo zal ik hen een vragenlijst voorleggen om te bepalen wat ze er zelf van vonden.



%---------- Verwachte resultaten ----------------------------------------------
\section{Verwachte resultaten}
\label{sec:verwachte_resultaten}

\begin{figure}[h]
    \centering
    \includegraphics[scale=0.5]{mockupGraph.JPG}
    \caption{Mock-up grafiek}
    \label{graph}
\end{figure}



%---------- Verwachte conclusies ----------------------------------------------
\section{Verwachte conclusies}
\label{sec:verwachte_conclusies}

Ik denk dat een dergelijke manier van revalideren zeker zal aanslaan bij de patiënten en het hen meer zal motiveren om hun oefeningen consistent uit te voeren. Ook denk ik dat de virtuele omgeving een positief effect zal hebben op de pijnprikkels en dat men sneller resultaten zal boeken op vlak van revalidatie. 



%%=============================================================================
%% Inleiding
%%=============================================================================

\chapter{Inleiding}
\label{ch:inleiding}

Virtual reality is een zeer hedendaags begrip dat de komende jaren al maar meer aandacht zal opeisen in ons dagelijkse leven. Zowel op vlak van entertainment als op business vlak zal VR ongetwijfeld zorgen voor grote doorbraken. De mogelijkheden ervan zijn dan ook (bijna) eindeloos en momenteel verre van allemaal opgeklaard.

Bij virtual reality wordt er een virtuele wereld gecreëerd voor de gebruiker die waarneembaar wordt door een zogenaamde VR bril of headset. Wat deze technologie de laatste jaren zo populair maakt is het ‘fully immersive’ effect. Dit effect zorgt ervoor dat de gebruikerservaring levensecht wordt. De gebruiker kan de virtuele wereld op een gelijke manier manipuleren als de echte wereld. Om dit te kunnen verzorgen is er natuurlijk een zeer grote hoeveelheid rekenkracht van een computer nodig die enkele jaren geleden nog niet haalbaar was.

Vele technologiebedrijven zagen de laatste jaren al het potentieel in deze nieuwe markt en brachten  een eigen VR headset uit. Zo kwam Google bijvoorbeeld met de ‘Cardboard’ en de ‘Daydream’, Samsung bracht dan weer de ‘Gear VR’ uit. Wat deze headsets gemeen hebben is dat ze niet afhankelijk zijn van een vaste computer maar van een smartphone. Dit in tegenstelling tot bedrijven zoals bijvoorbeeld Oculus of HTC die eerder al uitkwamen met headsets die wel nood hebben aan een vaste computer. Het voordeel hiervan is dat vaste computers (momenteel nog) een grotere rekenkracht hebben dan smartphones. Het ‘fully immersive’ effect is dus groter bij deze headsets en de kwaliteit van de gebruikerservaring neemt hier dus toe. 

Een belangrijke sector waar VR voor grote doorbraken zal zorgen is ongetwijfeld de gezondheidssector. Zo wordt deze technologie vandaag al gebruikt in tal van medische situaties. Door de toenemende populariteit zal het aanbod aan mogelijkheden dan ook enkel maar vergroten. In deze scriptie zal de focus vooral liggen op de interessante wereld van de revalidatie en de mogelijke voordelen die VR zou kunnen bieden.


\section{Probleemstelling}
\label{sec:probleemstelling}
Vele mensen consulteren op een gegeven moment in hun leven wel eens een kinesist. Al is het voor een sportblessure of een ouderdomsblessure. Iedereen die bij de kinesist langsgaat heeft uiteindelijk maar één hoofddoel en dat is revalideren, en dat liefst zo snel mogelijk. Hoewel een sessie bij de kinesist natuurlijk zeer opluchtend kan zijn is dit nog maar een klein deel van de werkelijke revalidatie.  Het andere grote deel zal men zelfstandig moeten doen door de oefeningen die men verkregen heeft uit te voeren. Bij velen loopt het hier dan ook wat fout. Zo zijn de repetitieve, soms wat saaie oefeningen niet voor iedereen motiverend genoeg om ze consistent uit te voeren. Dit heeft dan ook een negatieve invloed op het revalidatieproces.

Dit is waar virtual reality te hulp kan schieten. De technologie is de laatste jaren zeer geëvolueerd en heeft zeer veel mogelijkheden in de gezondheidszorg. Er zal dan ook onderzocht worden of het patiënten meer kan motiveren om hun oefeningen uit te voeren m.b.v. Virtual reality. Het zal dan ook rechtstreeks in verband gelegd kunnen worden met een efficiëntere revalidatie. 

Patiënten hebben soms zo een erge pijn dat ze pijnstillers moeten slikken voordat ze überhaupt hun oefeningen kunnen uitvoeren. Virtual reality zou deze trend ook kunnen inperken. De uitkomst van deze scriptie zou dan ook belangrijke veranderingen teweeg kunnen brengen in de sector.


\section{Onderzoeksvraag}
\label{sec:onderzoeksvraag}

In deze scriptie zullen enkele onderzoeksvragen behandeld worden:
\begin{itemize}
	\item Zal de patiënt zich meer gemotiveerd voelen bij het uitvoeren van zijn/haar oefeningen?
	\newline
	\item Zal het effect van VR een positieve invloed hebben op de pijngrens en vervolgens ook op het bewegingsbereik van de patiënt?
\end{itemize}

\section{Onderzoeksdoelstelling}
\label{sec:onderzoeksdoelstelling}
De onderzoeksdoelstelling van deze scriptie is nagaan of er potentieel zit in het gebruik van virtual reality in de revalidatie. De onderzoeksvragen van hierboven zullen alvast een criteria voor succes vormen.


\section{Opzet van deze bachelorproef}
\label{sec:opzet-bachelorproef}

% Het is gebruikelijk aan het einde van de inleiding een overzicht te
% geven van de opbouw van de rest van de tekst. Deze sectie bevat al een aanzet
% die je kan aanvullen/aanpassen in functie van je eigen tekst.

De rest van deze bachelorproef is als volgt opgebouwd:

In Hoofdstuk~\ref{ch:stand-van-zaken} wordt een overzicht gegeven van de stand van zaken binnen het onderzoeksdomein, op basis van een literatuurstudie.

In Hoofdstuk~\ref{ch:methodologie} wordt de methodologie toegelicht en worden de gebruikte onderzoekstechnieken besproken om een antwoord te kunnen formuleren op de onderzoeksvragen.

% TODO: Vul hier aan voor je eigen hoofstukken, één of twee zinnen per hoofdstuk

In Hoofdstuk~\ref{ch:conclusie}, tenslotte, wordt de conclusie gegeven en een antwoord geformuleerd op de onderzoeksvragen. Daarbij wordt ook een aanzet gegeven voor toekomstig onderzoek binnen dit domein.


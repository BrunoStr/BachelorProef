%%=============================================================================
%% Inleiding
%%=============================================================================

\chapter{Inleiding}
\label{ch:inleiding}

Virtual reality is een zeer hedendaags begrip dat de komende jaren al maar meer aandacht zal opeisen in ons dagelijkse leven. Zowel op vlak van entertainment als op business vlak zal VR ongetwijfeld zorgen voor grote doorbraken. De mogelijkheden ervan zijn dan ook (bijna) eindeloos en momenteel verre van allemaal opgeklaard.

Bij virtual reality wordt er een virtuele wereld gecreëerd voor de gebruiker die waarneembaar wordt door een zogenaamde VR bril of headset. Wat deze technologie de laatste jaren zo populair maakt is het ‘fully immersive’ effect. Dit effect zorgt ervoor dat de gebruikerservaring levensecht wordt. De gebruiker kan de virtuele wereld op een gelijke manier manipuleren als de echte wereld. Om dit te kunnen verzorgen is er natuurlijk een zeer grote hoeveelheid rekenkracht van een computer nodig die enkele jaren geleden nog niet haalbaar was.

Vele technologiebedrijven zagen de laatste jaren al het potentieel in deze nieuwe markt en brachten  een eigen VR headset uit. Zo kwam Google bijvoorbeeld met de ‘Cardboard’ en de ‘Daydream’, Samsung bracht dan weer de ‘Gear VR’ uit. Wat deze headsets gemeen hebben is dat ze niet afhankelijk zijn van een vaste computer maar van een smartphone. Dit in tegenstelling tot bedrijven zoals bijvoorbeeld Oculus of HTC die eerder al uitkwamen met headsets die wel nood hebben aan een vaste computer. Het voordeel hiervan is dat vaste computers (momenteel nog) een grotere rekenkracht hebben dan smartphones. Het ‘fully immersive’ effect is dus groter bij deze headsets en de kwaliteit van de gebruikerservaring neemt hier dus toe. 

Een belangrijke sector waar VR voor grote doorbraken zal zorgen is ongetwijfeld de gezondheidssector. Zo wordt deze technologie vandaag al gebruikt in tal van medische situaties. Door de toenemende populariteit zal het aanbod aan mogelijkheden dan ook enkel maar vergroten. In deze scriptie zal de focus vooral liggen op de interessante wereld van de revalidatie en de mogelijke voordelen die VR zou kunnen bieden.


\section{Probleemstelling}
\label{sec:probleemstelling}

Uit je probleemstelling moet duidelijk zijn dat je onderzoek een meerwaarde heeft voor een concrete doelgroep. De doelgroep moet goed gedefinieerd en afgelijnd zijn. Doelgroepen als ``bedrijven,'' ``KMO's,'' systeembeheerders, enz.~zijn nog te vaag. Als je een lijstje kan maken van de personen/organisaties die een meerwaarde zullen vinden in deze bachelorproef (dit is eigenlijk je steekproefkader), dan is dat een indicatie dat de doelgroep goed gedefinieerd is. Dit kan een enkel bedrijf zijn of zelfs één persoon (je co-promotor/opdrachtgever).

\section{Onderzoeksvraag}
\label{sec:onderzoeksvraag}

Wees zo concreet mogelijk bij het formuleren van je onderzoeksvraag. Een onderzoeksvraag is trouwens iets waar nog niemand op dit moment een antwoord heeft (voor zover je kan nagaan). Het opzoeken van bestaande informatie (bv. ``welke tools bestaan er voor deze toepassing?'') is dus geen onderzoeksvraag. Je kan de onderzoeksvraag verder specifiëren in deelvragen. Bv.~als je onderzoek gaat over performantiemetingen, dan 

\section{Onderzoeksdoelstelling}
\label{sec:onderzoeksdoelstelling}

Wat is het beoogde resultaat van je bachelorproef? Wat zijn de criteria voor succes? Beschrijf die zo concreet mogelijk.

\section{Opzet van deze bachelorproef}
\label{sec:opzet-bachelorproef}

% Het is gebruikelijk aan het einde van de inleiding een overzicht te
% geven van de opbouw van de rest van de tekst. Deze sectie bevat al een aanzet
% die je kan aanvullen/aanpassen in functie van je eigen tekst.

De rest van deze bachelorproef is als volgt opgebouwd:

In Hoofdstuk~\ref{ch:stand-van-zaken} wordt een overzicht gegeven van de stand van zaken binnen het onderzoeksdomein, op basis van een literatuurstudie.

In Hoofdstuk~\ref{ch:methodologie} wordt de methodologie toegelicht en worden de gebruikte onderzoekstechnieken besproken om een antwoord te kunnen formuleren op de onderzoeksvragen.

% TODO: Vul hier aan voor je eigen hoofstukken, één of twee zinnen per hoofdstuk

In Hoofdstuk~\ref{ch:conclusie}, tenslotte, wordt de conclusie gegeven en een antwoord geformuleerd op de onderzoeksvragen. Daarbij wordt ook een aanzet gegeven voor toekomstig onderzoek binnen dit domein.


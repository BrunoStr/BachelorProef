%%=============================================================================
%% Inleiding
%%=============================================================================

\chapter{Inleiding}
\label{ch:inleiding}

Virtual reality is een zeer hedendaags begrip dat de komende jaren al maar meer aandacht zal opeisen in ons dagelijkse leven. Zowel op vlak van entertainment, business en ook op medisch vlak zal VR ongetwijfeld nog voor grote doorbraken zorgen. De mogelijkheden ervan zijn dan ook (bijna) eindeloos en momenteel verre van allemaal opgeklaard.

Bij virtual reality wordt er een virtuele wereld gecreëerd voor de gebruiker die waarneembaar wordt door een zogenaamde VR bril of headset. Wat deze technologie de laatste jaren zo populair maakt is het ‘fully immersive’ effect dat mogelijk wordt gemaakt door de grote rekenkracht van de hedendaagse computers die enkele jaren geleden nog niet haalbaar was. Dit effect zorgt ervoor dat de gebruikerservaring levensecht wordt. De gebruiker kan de virtuele wereld op een manier manipuleren die gelijkt op de echte wereld. 

Grote technologiebedrijven zagen de laatste jaren het potentieel ervan al in en investeerden in deze nieuwe markt. Velen brachten dan ook al een eigen VR headset uit. Zo kwam internetgigant Google bijvoorbeeld met de ‘Cardboard’ en de ‘Daydream’ op de markt. Daarnaast kwam Samsung ook met een mobiele VR headset op de proppen, 'Gear VR’ genaamd. Wat deze headsets gemeen hebben is dat ze niet afhankelijk zijn van een vaste computer maar van een smartphone. Dit in tegenstelling tot bedrijven zoals bijvoorbeeld Oculus of Htc die eerder al uitkwamen met headsets die wel nood hebben aan een vaste computer. Het voordeel hiervan is dat vaste computers (momenteel nog) een grotere rekenkracht hebben dan smartphones waardoor de VR belevenis veel kwaliteitsvoller wordt dan op mobile. Omdat de instapkosten voor mobile VR echter veel lager liggen voor de consument dan een high-end desktop VR systeem, zijn bedrijven zoals Google en Samsung dan ook de grondleggers voor de VR hype die momenteel aan de gang is.


Een belangrijke sector waar VR ongetwijfeld nog voor grote doorbraken zal zorgen is de gezondheidssector. Zo wordt deze technologie vandaag al gebruikt in tal van medische situaties. Door de toenemende populariteit zal het aanbod aan mogelijkheden dan ook enkel maar vergroten. In deze scriptie zal de focus vooral liggen op de interessante wereld van de revalidatie van het bovenste lidmaat en de mogelijke voordelen die VR zou kunnen bieden.


\section{Probleemstelling}
\label{sec:probleemstelling}
Vele mensen consulteren op een gegeven moment in hun leven wel eens een kinesist. Al is het voor een sportblessure, ouderdomsblessure of dergelijke. Iedereen die bij de kinesist langsgaat heeft uiteindelijk maar één hoofddoel: dat is natuurlijk revalideren, en dat liefst zo snel mogelijk. Hoewel een sessie bij de kinesist natuurlijk zeer opluchtend kan zijn is dit nog maar een klein deel van de werkelijke revalidatie. Het andere grote deel zal men zelfstandig moeten doen door de oefeningen die men verkregen heeft uit te voeren. Bij velen loopt het hier soms dan ook fout. Zo zijn de oefeningen voor bepaalde mensen soms wat te repetitief en te saai. Hierdoor raken ze dan ook gedemotiveerd om ze consistent uit te blijven voeren. Dit heeft natuurlijk ook een negatieve invloed op het revalidatieproces. 

Dit is waar virtual reality te hulp kan schieten. De technologie is de laatste jaren zeer geëvolueerd en heeft al laten blijken dat het zeer veel mogelijkheden in de gezondheidszorg kan bieden. In deze scriptie zal er dan ook onderzocht worden of het patiënten eventueel meer kan motiveren om hun oefeningen uit te voeren. Het zou dan ook rechtstreeks in verband gesteld kunnen worden met een efficiëntere revalidatie.

Patiënten ervaren bij het uitvoeren van de oefeningen vaak nog hevige pijnprikkels. Deze pijnervaring wordt daarnaast vaak ook nog versterkt puur door psychologische redenen. Sommige patiënten dienen, voor ze überhaupt hun oefeningen uit kunnen voeren, eerst nog medicatie te slikken om deze pijnprikkels te onderdrukken. Ook hier zou virtual reality mogelijks een positieve invloed kunnen hebben aangezien de gebruiker wordt afgeleid van de échte realiteit en dus ook van de pijnprikkels. Ook dit aspect zal hier onderzocht worden. De uitkomst van deze scriptie zou dan ook een aanzet kunnen zijn tot fundamentele veranderingen in de sector.


\section{Onderzoeksvraag}
\label{sec:onderzoeksvraag}

In deze scriptie zullen enkele onderzoeksvragen behandeld worden:

- Zal de patiënt zich meer gemotiveerd voelen bij het uitvoeren van zijn oefeningen met behulp van virtual reality?

- Zal het effect van virtual reality een positieve invloed hebben op de pijnervaring van de patiënt?


\section{Onderzoeksdoelstelling}
\label{sec:onderzoeksdoelstelling}
De onderzoeksdoelstelling van deze scriptie is nagaan of er potentieel zit in het gebruik van virtual reality in de revalidatie en ook nagaan of er vanuit de sector zelf interesse zou zijn hiervoor. De onderzoeksvragen van hierboven zullen alvast een criteria voor succes vormen.


\section{Opzet van deze bachelorproef}
\label{sec:opzet-bachelorproef}

% Het is gebruikelijk aan het einde van de inleiding een overzicht te
% geven van de opbouw van de rest van de tekst. Deze sectie bevat al een aanzet
% die je kan aanvullen/aanpassen in functie van je eigen tekst.

De rest van deze bachelorproef is als volgt opgebouwd:

In Hoofdstuk 2 wordt een overzicht gegeven van de stand van zaken binnen het onderzoeksdomein, op basis van een literatuurstudie. Hier wordt de nadruk gelegd op de virtual reality en het technische aspect ervan.

In Hoofdstuk 3 wordt er opnieuw een overzicht gegeven van de stand van zaken binnen het onderzoeksdomein, op basis van een literatuurstudie. De nadruk wordt hier gelegd op virtual reality in de gezondheidszorg.

In Hoofdstuk 4 wordt het product besproken dat gerealiseerd werd om het experiment uit te voeren. Daarnaast wordt er ook besproken welke technologieën gebruikt werden om dit product te realiseren en welke features er in de toekomst nog geïmplementeerd kunnen worden.

In Hoofdstuk~\ref{ch:methodologie} wordt de methodologie toegelicht en worden de gebruikte onderzoekstechnieken besproken om een antwoord te kunnen formuleren op de onderzoeksvragen.

In Hoofdstuk 6 wordt het experiment en de resultaten ervan besproken.

In Hoofdstuk~\ref{ch:conclusie} wordt de conclusie gegeven en een antwoord geformuleerd op de onderzoeksvragen. Daarbij wordt ook een aanzet gegeven voor toekomstig onderzoek binnen dit domein.


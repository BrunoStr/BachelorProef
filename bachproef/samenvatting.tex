%%=============================================================================
%% Samenvatting
%%=============================================================================

% TODO: De "abstract" of samenvatting is een kernachtige (~ 1 blz. voor een
% thesis) synthese van het document.
%
% Deze aspecten moeten zeker aan bod komen:
% - Context: waarom is dit werk belangrijk?
% - Nood: waarom moest dit onderzocht worden?
% - Taak: wat heb je precies gedaan?
% - Object: wat staat in dit document geschreven?
% - Resultaat: wat was het resultaat?
% - Conclusie: wat is/zijn de belangrijkste conclusie(s)?
% - Perspectief: blijven er nog vragen open die in de toekomst nog kunnen
%    onderzocht worden? Wat is een mogelijk vervolg voor jouw onderzoek?
%
% LET OP! Een samenvatting is GEEN voorwoord!

%%---------- Nederlandse samenvatting -----------------------------------------
%
% TODO: Als je je bachelorproef in het Engels schrijft, moet je eerst een
% Nederlandse samenvatting invoegen. Haal daarvoor onderstaande code uit
% commentaar.
% Wie zijn bachelorproef in het Nederlands schrijft, kan dit negeren, de inhoud
% wordt niet in het document ingevoegd.

\IfLanguageName{english}{%
\selectlanguage{dutch}
\chapter*{Samenvatting}
\lipsum[1-4]
\selectlanguage{english}
}{}

%%---------- Samenvatting -----------------------------------------------------
% De samenvatting in de hoofdtaal van het document

\chapter*{\IfLanguageName{dutch}{Samenvatting}{Abstract}}

In deze scriptie wordt er onderzocht of er potentieel zit in het gebruik van virtual reality tijdens de revalidatie. Revalidatie is een belangrijk proces waar iedereen wel eens mee geconfronteerd wordt. Velen zijn echter niet altijd even gemotiveerd om repetitieve revalidatie oefeningen uit te voeren. De oefeningen worden dan ook vaak eens overgeslagen, wat dus zeker een invloed heeft op de duur van het revalidatietraject.
Ook wordt er onderzocht of men met behulp van virtual reality de pijntolerantie van een patiënt kan verhogen.

Hier wordt dus een zeker hedendaags probleem aangekaard waar een technologie zoals VR mogelijks enkele voordelen teweeg kan brengen.

Tijdens dit onderzoek werd er een applicatie ontwikkeld voor een VR headset. Hierbij kwam de gebruiker terecht in een natuurlijke omgeving waar hij appels kon plukken en in de juiste mand kon leggen. Bij het plukken van de appels moest er wel rekening gehouden worden met de op voorhand afgesproken beweging.

Personen die deelnamen aan het experiment dienden eerst enkele oefeningen zonder VR headset uit te voeren en hierna enkele met VR headset. Tussen de oefensessies door werden hen enkele vragenlijsten voorgeschoteld. Er waren 25 testpersonen, tussen de 19 en 85 jaar, die deelnamen aan het experiment. Bij hen waren er geen letsels vast te stellen, daarom werd er gebruik gemaakt van een wasknijper om een pijnprikkel te simuleren.

De resultaten die uit het onderzoek kwamen waren zeer positief. Het merendeel van de testpersonen merkte wel degelijk een verschil in pijn tijdens het gebruik van virtual reality. Daarnaast gaven ook vele personen aan dat ze zich meer gemotiveerd voelden wanneer ze gebruik konden maken van het VR systeem. Deze technologie wekte bij de testpersonen meteen interesse en na het experiment zagen velen van hen ook het potentieel in het gebruik tijdens de revalidatie ervan in.

Verder onderzoek zou dus zeker mogelijk zijn omtrent dit onderwerp. Zo zouden er bij een gesofisticeerder VR systeem nog significantere verschillen in pijnervaring kunnen optreden. Ook zou het zeer interessant zijn om de applicatie op personen te testen met werkelijke letsels, onder begeleiding van de kinesist.

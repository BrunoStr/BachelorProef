%%=============================================================================
%% Voorwoord
%%=============================================================================

\chapter*{Voorwoord}
\label{ch:voorwoord}

%% TODO:
%% Het voorwoord is het enige deel van de bachelorproef waar je vanuit je
%% eigen standpunt (``ik-vorm'') mag schrijven. Je kan hier bv. motiveren
%% waarom jij het onderwerp wil bespreken.
%% Vergeet ook niet te bedanken wie je geholpen/gesteund/... heeft

Voor u ligt de scriptie `Virtual reality als sleutel tot succesvolle revalidatie`. In dit onderzoek worden de potentiele voordelen van het gebruik van virtual reality in de revalidatie onderzocht. De resultaten werden verzameld aan de hand van 25 testpersonen tussen de 19 en 85 jaar. Deze personen dienden aan de hand van een zelfontwikkelde VR applicatie enkele oefeningen uit te voeren en hierover meerdere enquêtes in te vullen. Ik zou alle mensen die deelnamen aan dit experiment dan ook uitdrukkelijk willen bedanken.

Tijdens het verwezenlijken van deze scriptie heb ik ook op enkele helpende handen kunnen rekenen. Eerst en vooral zou ik graag mijn promotor, Jens Buysse, willen bedanken. Hij was vanaf het begin zeer geïnteresseerd in mijn onderzoek wat me zeer motiveerde om dit effectief te verwezenlijken. Ook organiseerde hij regelmatig samenkomsten waar we in groep de plannen op tafel legden en bespraken.

Daarnaast zou ik ook graag mijn co-promotor willen bedanken, Vincent Lejeune. Hij stond me bij tijdens dit onderzoek op vlak van kinesitherapie. Hij zag meteen ook potentieel in mijn onderzoek en was zeer geïnteresseerd in de resultaten. Ook begeleidde hij mij bij het bedenken van een concept voor het prototype van de VR game.

Tot slot zou ik nog mijn medestudent, Casper Verswijvelt, willen bedanken voor het uitlenen van zijn Oculus Go waardoor dit experiment mogelijk werd.

 Deze scriptie is geschreven in het kader van mijn afstuderen aan de opleiding toegepaste informatica aan de Hogeschool Gent.
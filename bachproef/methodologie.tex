\chapter{Applicatie}
\label{ch:applicatie}

\section{Gerealiseerde product}
Het gerealiseerde product is een game waar de gebruiker in een virtuele omgeving terecht komt. Hier kan hij met behulp van de controllers fictieve appels plukken uit bomen en ze in de juiste mand leggen. Bij het uitschrijven van het idee voor de applicatie werd er vooral gefocust op één bepaalde beweging waar men van beneden gecontroleerd naar boven dient te reiken en opnieuw naar beneden. De applicatie richt zich dan vooral ook op de revalidatie van de bovenste ledematen. Op \textbf{figuur 4.1}, \textbf{figuur 4.2}, \textbf{figuur 4.3} en \textbf{figuur 4.4} kan u hiervan een visualisatie terugvinden.

\section{Gebruikte Software}
Het product werd gerealiseerd met behulp van de game engine Unity3D. De rede waarom er voor deze engine gekozen werd is omdat de integratie met de Oculus Go zeer goed verwerkt zit in het programma. Na het downloaden van de Oculus Integration package kan men vrijwel meteen aan de slag ermee. Ook werd er gebruik gemaakt van de Unity Asset store. Deze store bevat heel wat (gratis) 3D modellen, geluidseffecten, animaties, ...
Unity maakt standaard ook gebruik van Visual Studio voor het schrijven van scripts.

\section{Mogelijke features voor in de toekomst}
Momenteel is de applicatie nog puur een proof of concept maar deze zou in de toekomst eventueel nog verder uitgewerkt kunnen worden. Men zou bijvoorbeeld verschillende levels kunnen toevoegen die elk focussen op een andere beweging. Voorlopig is er ook nog geen score-systeem wat het spel toch zeker uitdagender zou maken. Daarnaast zou het ook zeer interessant zijn voor de kinesist om wat gegevens op te kunnen vragen van de patiënt of hij zijn bewegingen heeft gedaan, of er vorderingen zijn, ...

\begin{figure}[h]
    \centering
    \includegraphics[scale=0.5]{apple1.png}
    \caption{ApplePicker screenshot welkomstbord}
\end{figure}

\begin{figure}[h]
    \centering
    \includegraphics[scale=0.5]{apple2.png}
    \caption{ApplePicker screenshot appels in de boom}
\end{figure}

\begin{figure}[h]
    \centering
    \includegraphics[scale=0.5]{apple3.png}
    \caption{ApplePicker screenshot appel plukken}
\end{figure}

\begin{figure}[h]
    \centering
    \includegraphics[scale=0.5]{apple4.png}
    \caption{ApplePicker screenshot appels in de manden}
\end{figure}

%%=============================================================================
%% Methodologie
%%=============================================================================

\chapter{Methodologie}
\label{ch:methodologie}

%% TODO: Hoe ben je te werk gegaan? Verdeel je onderzoek in grote fasen, en
%% licht in elke fase toe welke stappen je gevolgd hebt. Verantwoord waarom je
%% op deze manier te werk gegaan bent. Je moet kunnen aantonen dat je de best
%% mogelijke manier toegepast hebt om een antwoord te vinden op de
%% onderzoeksvraag.

Om de gestelde onderzoeksvragen te beantwoorden werd er een experiment uitgevoerd tijdens dit onderzoek.
Hoe dit experiment werd opgesteld wordt hieronder nader verklaard.

\section{Plan van aanpak}

Het experiment werd grotendeels uitgevoerd op willekeurige testpersonen. Bij deze testpersonen werd er a.d.h.v. een wasknijper een pijnprikkel opgewekt. Deze pijnprikkel moest gedurende enkele oefensessies verdragen worden.

Bij een eerste oefensessie moest de persoon tien maal een beweging uitvoeren met de wasknijper op de vinger. 
Tijdens de tweede oefensessie kreeg de persoon de VR bril en de hoofdtelefoon opgezet. Hier kwam hij terecht in 'ApplePicker', het gerealiseerde prototype voor dit onderzoek. In deze game moest de persoon opnieuw tien maal de beweging uitvoeren, met de wasknijper op de vinger.

Na elke oefensessie diende de persoon een vragenlijst in te vullen 
over zaken zoals: hoe de pijnervaring was tijdens de oefeningen, of de persoon zich altijd even gemotiveerd voelt om revalidatie oefeningen uit te voeren, ...

De data uit deze vragenlijsten werd hierna verwerkt en omgezet in grafieken die de onderzoeksvragen konden beantwoorden.

\chapter{Experiment}

\section{Materiaal}
Om het experiment uit te voeren waren enkele zaken onmisbaar:

- VR headset en controller: Hier werd de Oculus Go gebruikt. Dit is een standalone VR headset dat beschikt over \footnote{3 degrees of freedom. Enkel de rotatie wordt getrackt, de positie in de ruimte niet.}3DOF.

- Hoofdtelefoon: Dit stimuleert het surround sound effect, bedoelt om de persoon af te leiden van de alledaagse geluiden rondom.

- Computer: Hierop werden de vragenlijsten ingevuld na de oefensessies. Ook kon de kinesist/begeleider hier de bewegingen van de patiënt meevolgen.

- Polsgewichten (eventueel): Deze konden bij de patiënt omgedaan worden om de intensiteit van de oefeningen te vergroten.

\newpage
\subsection{Testpersonen}
Als testpersonen werden er 25 willekeurige personen gekozen tussen de 19 en 85 jaar. Op \textbf{figuur 6.1} kan u bijhorende boxplot terugvinden die een beeld geeft van de verdeling van de leeftijden per geslacht. De testpersoon werd op voorhand niet ingelicht over de opzet van het onderzoek. Dit om de onbevooroordeeldheid zo veel mogelijk te beperken.

\begin{figure}[h]
    \centering
    \includegraphics[scale=0.7]{Boxplot_Leeftijd.png}
    \caption{Boxplot van leeftijden van proefpersonen}
    \label{figuur 6.1}
\end{figure}

Gedurende het experiment werd de proefpersoon een wasknijper opgedaan om een pijnprikkel op te wekken. Hierna diende de oefening uitgevoerd te worden (\textbf{figuur 6.2}). 

\begin{figure}[h]
    \centering
    \includegraphics[scale=0.8]{luc.png}
    \caption{Proefpersoon test de applicatie met wasknijper op de vinger en voert de beweging uit}
    \label{figuur 6.2}
\end{figure}

\newpage

\section{Resultaten}
Iedere persoon die zich onderwierp aan het experiment diende enkele vragenlijsten in te vullen. Onderstaande grafieken werden uit deze data geëxtraheerd.
\footnote{https://github.com/BrunoStr/Bacherlorproef\_scripts/blob/master/R\_script}De uitwerking van de scripts om onderstaande grafieken te bekomen is terug te vinden op Github.

\begin{figure}[h]
    \centering
    \includegraphics[scale=0.6]{barplot_motivatie.png}
    \caption{In welke mate men gemotiveerd is om consistent de revalidatie oefeningen uit te voeren}
    \label{figuur 6.3}
\end{figure}

Bovenstaande grafiek (\textbf{figuur 6.3}) toont duidelijk dat mensen zich niet altijd gemotiveerd voelen om hun revalidatie oefeningen uit te voeren. Zo blijkt dat slechts 36\% van de personen zich altijd gemotiveerd voelt en 16\% zich nooit echt gemotiveerd voelt. 48\% voelt zich dan weer soms gemotiveerd om hun oefeningen uit te voeren.

\begin{figure}[h]
    \centering
    \includegraphics[scale=0.65]{barplot_overslaan.png}
    \caption{In welke mate oefensessies worden overgeslagen}
    \label{figuur 6.4}
\end{figure}

Door het gebrek aan motivatie dat bij sommmigen wel eens de kop op steekt durft men dan ook hier en daar eens een oefensessie over te slaan (\textbf{figuur 6.4}). 28\% geeft toe dat ze af en toe hun oefeningen overslaan, 44\% soms wel eens en 12\% zegt ze vaak over te slagen. Hiernaast zijn er ook enkele zeer gedreven mensen die nooit een sessie overslaan (16\%).

\begin{figure}[h]
    \centering
    \includegraphics[scale=0.7]{boxplot_PijnZonder.png}
    \caption{Pijnervaring testgebruikers bij oefensesie zonder VR}
    \label{figuur 6.5}
\end{figure}

Uit \textbf{figuur 6.5} kan men een mediaan van 5, een interkwartielafstand van 2 en een spreidingbreedte van 6 vaststellen.

\begin{figure}[h]
    \centering
    \includegraphics[scale=0.7]{boxplot_PijnMet.png}
    \caption{Pijnervaring testgebruikers bij oefensesie met VR}
    \label{figuur 6.6}
\end{figure}

Uit \textbf{figuur 6.6} kan men een mediaan van 3, een interkwartielafstand van 2 en een spreidingsbreedte van 7 vaststellen.

Hier kan men dus concluderen dat op een paar uitschieters na, de pijnervaring bij iedereen significant gedaald was. Wanneer men de gemiddelde afname in pijn berekent komt men uit op een waarde van 1.88 op een schaal van 10 ofwel 18.8\%.

\begin{figure}[h]
    \centering
    \includegraphics[scale=0.7]{barplot_verschilPijn.png}
    \caption{Diagram van verschil in pijnervaring die de proefpersonen ondervonden met en zonder VR}
    \label{figuur 6.7}
\end{figure}

\newpage

Opnieuw wordt er in \textbf{figuur 6.7} bevestigd dat er wel degelijk een verschil in pijn werd waargenomen bij de testpersonen. Zo'n 72\% zegt dat er inderdaad een verschil was in pijn wanneer men de oefening met VR uitvoerde. 28\% van de mensen merkte echter slechts deels een verschil.

\begin{figure}[h]
    \centering
    \includegraphics[scale=0.7]{barplot_profVr.png}
    \caption{Diagram van personen die denken dat professionelere VR meer invloed zou hebben op pijnervaring}
    \label{figuur 6.8}
\end{figure}

In de grafiek op \textbf{figuur 6.8} merkt men dat velen ervan overtuigd waren dat het verschil in pijn nog significanter zou zijn wanneer men gebruik zou maken van gesofisticeerdere VR systemen (vrij bewegen in de ruimte, dynamischere omgeving, hogere kwaliteit, ...). 80\% van de testpersonen stemde hier volop mee in en 20\% zag de mogelijkheid ervan in.

\begin{figure}[h]
    \centering
    \includegraphics[scale=0.65]{barplot_extraMotivatie.png}
    \caption{Diagram van personen die geloven dat VR voor extra motivatie kan zorgen}
    \label{figuur 6.9}
\end{figure}

Op \textbf{figuur 6.9} ziet men dat een groot aandeel zich meer gemotiveerd zou voelen om zijn oefeningen uit te voeren wanneer ze toegang zouden hebben tot een VR systeem. Opnieuw 80\% van de testpersonen zou zich meer gemotiveerd voelen, 16\% zou zich misschien meer gemotiveerd voelen en 4\% helemaal niet.

\begin{figure}[h]
    \centering
    \includegraphics[scale=0.65]{barplot_potentieel.png}
    \caption{Diagram van personen die geloven dat VR potentieel heeft in de revalidatie}
    \label{figuur 6.10}
\end{figure}

Tenslotte kan men uit \textbf{figuur 6.10} vaststellen dat er 92\% van de testpersonen potentieel zagen in het gebruik van VR tijdens de revalidatie. 8\% van de personen had hier echter nog wat zijn twijfels bij.


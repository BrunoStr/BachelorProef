%%=============================================================================
%% Conclusie
%%=============================================================================

\chapter{Conclusie}
\label{ch:conclusie}

% TODO: Trek een duidelijke conclusie, in de vorm van een antwoord op de
% onderzoeksvra(a)g(en). Wat was jouw bijdrage aan het onderzoeksdomein en
% hoe biedt dit meerwaarde aan het vakgebied/doelgroep? 
% Reflecteer kritisch over het resultaat. In Engelse teksten wordt deze sectie
% ``Discussion'' genoemd. Had je deze uitkomst verwacht? Zijn er zaken die nog
% niet duidelijk zijn?
% Heeft het onderzoek geleid tot nieuwe vragen die uitnodigen tot verder 
%onderzoek?

In deze scriptie werd onderzocht of met behulp van virtual reality de pijnervaring van een patiënt onderdrukt kan worden. Daarnaast werd er geanalyseerd of dergelijke technologie mogelijks ook de motivatie van patiënten tijdens de revalidatie kon stimuleren.

De onderzoeksresultaten bevestigen dat vele patiënten zich maar af en toe tot zelfs nooit gemotiveerd voelen om revalidatie oefeningen uit te voeren. Hierdoor slaan ze dan ook geregeld oefensessies over wat het revalidatietraject niet ten goede komt.

Op basis van uitgevoerde experimenten kan er geconcludeerd worden dat de patiënt effectief een verminderde pijnervaring beleefd bij het uitvoeren van zijn revalidatie oefeningen met behulp van virtual reality. Het merendeel van de respondenten geloofde dat deze vermindering in pijn nog beduidender zou kunnen zijn bij het gebruik van gesofisticeerdere VR systemen.

In de context van deze scriptie werd er bij de testpersonen een pijnprikkel opgewekt. Het zou zeker interessant zijn voor mogelijks verdere onderzoeken om het effect op daadwerkelijke patiënten te bestuderen.

Ook werd er bij velen geconstateerd dat zij zich meer gemotiveerd voelden bij het uitvoeren van de oefeningen wanneer zij een concreet doel voor ogen kregen, zoals een dergelijke VR game.

Deze scriptie legt dus zeker een aantal significante voordelen van virtual reality in de wereld van de revalidatie bloot. Daarnaast bleek ook uit de experimenten dat patiënten zeker openstaan voor het gebruik van VR tijdens hun revalidatie en dat ze het potentieel ervan zeker inzien.

Momenteel is de prijsdrempel echter voor particulier gebruik nog wat te hoog. Voor kinesisten is het zeker al overweegbaar om de stap te zetten. In de nabije toekomst zal deze drempel gegarandeerd enkel maar slinken waardoor de technologie toegankelijker dan ooit zal worden.

